In this chapter, concrete usage examples are given to highlight the practical nature of the system.
Simon, Niklas, Jona and Meike will be normal users of the system, whereas Timo will be the server administrator with access to the admin-account.

\subsection{Login}\label{subsec:us-login}
Simon and Niklas are avid readers of a german newspaper.
To share their articles with the rest of the family, they create accounts (\ref{subsubsec:signup}): Simon uses the web-interface, Niklas creates the account from inside the browser-extension.

\subsection{Creating, sharing and accessing a replic}\label{subsec:us-creating-and-sharing-a-replic}
Simon reads an article that, he believes, his Brother Jona will also find interesting.
While the tab for the article is opened on his browser, he uses the extension, in which he is logged-in, to create a replic (\ref{subsubsec:create-replic}).
He sets an expiration date for one week.
After the replic is created, he copies the link to it (\ref{subsubsec:copy-replic-links}) and texts it to Jona. \newline
Jona receives the link via the messenger from Simon and clicks it.
His browser opens (\ref{subsubsec:select-replic}) and shows the replic overview (\ref{subsubsec:view-replic}), where he reads the description of the replic that Simon entered.
He proceeds to open the link and read the article.

\subsection{Accesing own replics}\label{subsec:us-accesing-own-replics}
After Jona liked the article so much, Simon is inclined to also share the article with Niklas.
So he doesn't have to create a new replic, he visits the replics view, goes to the replic screen for the only replic he owns, copies the link (\ref{subsubsec:copy-replic-links}) and sends it to Niklas.

\subsection{Accessing replic too late}\label{subsec:us-accessing-replic-too-late}
Because Niklas is busy, he only receives the link after a few days.
When he tries to access the replic, he cannot view it, due to the expiration date being met.

\subsection{Reporting a replic}\label{subsec:us-reporting-a-replic}
An anonymous user has created a replic and sent the link to Jona via E-mail.
After observing the replic (\ref{subsubsec:view-replic}), Jona realizes that the content is not appropriate.
He creates a report for the replic (\ref{subsubsec:report-replic}).

\subsection{Reviewing a report}\label{subsec:us-reviewing-a-report}
During hsi weekly admin-tour, Timo spots the report filed by Jona.
After checking the replic, he comes to the conclusion that the replic is not appropriate, reviews the report (\ref{subsubsec:review-report}) and removes the replic (\ref{subsubsec:remove-replic}).

\subsection{Creating and deactivating an account (Admin)}\label{subsec:us-creating-an-account-admin}
To encourage Meike to join the system, Timo creates an account for her on the admin-panel (\ref{subsubsec:create-user}) and shares the credentials with her.
However, Meike prefers to read newspapers in paper, which causes Timo to deactivate her account (\ref{subsubsec:deactivate-acc-admin}).

\subsection{Resetting password}\label{subsec:resetting-password}
Sadly, Simon is very occupied with other aspects of his life and doesn't use replic-read often.
He forgets his password.
To help him, Timo resets his password (\ref{subsubsec:reset-pass}) to allow him to login (\ref{subsubsec:login}) again.

\subsection{Changing account data}\label{subsec:changing-account-data}
After some time of not using the system, Simon is embarassed by the username, email and color he set up when he was just a small child.
To fix this, he changes his username (\ref{subsubsec:change-username}) and profile color (\ref{subsubsec:change-color}).
Additionally, he requests to change his email to his new, less embarassing, one (\ref{subsubsec:change-email}) and verifies it by clicking the link in the email that was sent to him.